\section{*9.Uveďte a popište metody modifikace kódů.}
Nejběžnější metody modifikace kódů jsou: Zkracování kódů, rozšiřování kódů, děrování kódu, zvětšení kódu, zmenšení kódu, kombinování kódů a zřetězené kódy.
\begin{itemize}
    \item \textbf{Zkracování} - Princip zkrácení kódu spočívá ve zmenšení počtu využívaných kódových kombinací. Prvních $s$ prvků původního kódu je nulových a nepřenáší se - odstranění informačních prvků.
    \item \textbf{Rozšiřování} - Obecně rozšíření kódu spočívá v přidání $e$ zabezpečovacích prvků. Nejčastějším způsobem rozšiřování kódu je přidání paritního prvku
    \item \textbf{Děrování} - Děrování spočívá v odstraňování zabezpečovacích prvků, v Turbo kódech.
    \item \textbf{Zvětšení a zmenšení kódu} -  Vedou k nelineárním kódům. Zmenšování (zvětšování) kódu spočívá v odstranění (přidání) některých kódových slov z kódu.
    \item \textbf{Kombinování} - např. sčítat kódové kombinace různých kódů
    \item \textbf{Zřetězení} - také kombinování; výsledný zabezpečený datový tok z prvního kodéru je následně ještě jednou zabezpečen dalším kódem.
\end{itemize}

\section{*9.Co je to prokládaní a proč se používá? Jaké znáte metody prokládaní?}
Je to metoda, která umožňuje změnou pořadí prvků dosáhnout změnu typu chyb ze shlukových na nezávislé. V minulosti hlavně u přenosu televizního obrazu z důvodu veliké potřebné šířky pásma (blikání). Nejběžnější \textbf{metody} prokládání jsou: Konvoluční prokládání a blokové prokládání.

\section{*9.Co jsou to zřetězené kódy? Jaké dva základní typy zřetězení existují a jak se označují? Která technika se často využívá při zřetězení kódů?}
\begin{itemize}
    \item výsledný zabezpečený datový tok z prvního kodéru je následně ještě jednou zabezpečen dalším kódem
    \item paralelní zřetězení (turbo kódy) a sériové zřetězení - násobné kódy (product code)
    \item  Metoda se často kombinuje s prokládáním. Pro dekódování se vyžívaná modifikovaná Viterbiho metoda.
\end{itemize}
