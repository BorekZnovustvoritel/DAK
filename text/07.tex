\section{*7. Čím se liší cyklický a blokový kód? Uveďte některé cyklicky definované kódy.}
Cyklický kód je určen namísto vytvářecí matice G vytvářecím mnohočlenem G(x).
Vytvářecí matice G některých lineárních kódů jdou převést součty mezi řádky do takového tvaru, že každý řádek obsahuje stejné prvky, pouze o jedno místo posunuté v jednom směru.

CRC kódy, cyklické kódy odvozené z Hammingova kódu, Fireův kód 

\section{*7. Vysvětlete pojem Galoisovo těleso a uveďte, které kódy jej využívají?}
Pomocí prvků tohoto tělesa se uskutečňují operace spojené se zabezpečováním posloupnosti nezabezpečených signálových prvků. Galoisova tělesa GF (p$^r$). Zde $p$ je základ číselné soustavy (2) a $r$ odpovídá počtu zabezpečovacích prvků v mnohočlenu zabezpečené zprávy. Je tvořeno konečným počtem prvků n = p$^r$  a vzniklo rozšířením tzv. konečného tělesa Zp.

BCH kódy, Reed-Solomonovy (RS) kódy
